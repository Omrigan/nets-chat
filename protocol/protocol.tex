\documentclass[a4paper, 10pt]{article}

\usepackage[utf8]{inputenc}  
\usepackage{/home/omrigan/study/olegstyles}

\title{Протокол системы мгонвенного обмена сообщениями Elegram}
\date{\today}
\author{Oleg Vasilev}
\usepackage{listings}

\begin{document}
\maketitle

Протокол обеспечивает взаимодействие нескольких клиентов и сервера. Каждый клиент пишет сообщения, которые попадают в общий чат, и отправляются всем пользователям. 
В силу специфики задачи протокол должен поддерживать следующие операции:
\begin{enumerate}
\item Регистрация
\item Авторизация
\item Отправка сообщений
\item Окончание сессии
\end{enumerate} 
Протокол должен обрабатывать следующие ошибочные ситуации:
\begin{enumerate}
\item Ситуация повторой регистрации уже существующего пользователя 
\item Ситуация неверно введенного пароля
\end{enumerate}

\section{Команды от Клиента к Серверу}
\begin{tabular}{ll}
REGISTER <login> <password> & Регистрирует нового пользователя с указанным логином и паролем \\
LOGIN <login> <password> & Позволяет клиенту войти в аккаунт с указанным логином и паролем \\
LOGOUT &  Заканчивает сессию. Клиент должен авторизоваться снова\\
QUIT & Закрывает соединение \\
<msg> & Отправляет текст в чат \\
LISTALL & Перечислить всех зарегистиррованных пользователей. \\ 
LISTONLINE & Перечислить всех подключенных пользователей. \\ 
\end{tabular}
\\
LISTALL и LISTONLINE доступны только пользователю с логином admin. 
\section{Ответы от Сервера Клиенту}
\begin{tabular}{ll}
Login and password is required & Неправильное колличество аргументов у команд LOGIN и REGISTER \\ 
No such account & Авторизация с несуществующим аккаунтом \\ 
User already exists & Такой пользователь уже существует \\ 
\end{tabular}

\section{Коррекная операция "Всем привет в этом чатике"}
\texttt{ 
    \\
    --> REGISTER oleg oleg \# Регистрация пользователя\\
    <-- Registered \# Регистрация успешна \\
    --> Hello everyone from this chatik \# Отправка сообщения \\
    <-- oleg: Hello everyone from this chatik \# Получение только что отправленного сообщения\\
}
\newpage
\section{Обработка ошибочной ситуации "Авторизация с неправильным паролем"}
\texttt{ \\
    --> REGISTER oleg oleg \# Регистрация пользователя\\
    <-- Registered \# Регистрация успешна \\
    --> LOGOUT \# Окончание сессии \\
    <-- Logged out \# Поддверждение окончания сессии \\
    --> LOGIN oleg ivan \# Авторизация с неправильным паролем \\
    <-- Wrong password \# Неправильный пароль
}


\end{document}
